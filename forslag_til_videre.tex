\chapter{Forslag til videre arbejde med 3D scanner}
Til videre arbejde med 3D scanner vil det være en stor fordel at have en mere fast opstilling en tripod. Et forslag kunne være at bygge en mørklagt kasse hvor der er plads til både scanningsudstyr samt prøveemne. Dette vil være med til at beskytte og sikre opstillingen, da den bliver langt mere stabil.

Ved videre arbejde med scanning af permafrost kerner anbefales der en nærmere undersøgelse af hvor stor rotationsvinkel der kan anvendes uden at det går udover resultatet. De prøver som er skannet i dette projekt er forholdsvis ensformige, og har glat overflade, det ville derfor været interessant at kikke på hvordan scanneren takler en mere ujævn overflade med f.eks. sprækker. 
Derudover anbefales der at kikke på farveoptimering, hvordan den meshede prøves farve bliver identisk med prøvens faktiske farve. 
Andre parametre der bør undersøges for indvirkning på resultat er triangulerings-afstanden og valg af mesh.

Til sidst nævnes 3D- moddelering i pdf-format, modellerne af prøverne kan eksporteres til pdf-format, dog er det ikke lykkedes i denne rapport at få billede lagt rundt meshet, hvilket vil have stor betydning for præsentation af resultaterne.