\chapter{Konklusion}
Baseret på de opnåede resultater for klassifikation og bulkdensitet ses der ingen væsentlig forskel på bulkdensitets beregningerne for archimedes princip og 3D scanning, hverken i forhold til rotationsvinkel eller mængden af is på prøvens overflade, dog har testscanning på dtu compute vist at hvis isindholdet bliver højt nok og isen klar nok så kan ikke struktureret lys metoden indsamle data om overfladen. Det kan derfor ikke udelukkes at der vil være enkelte prøver eller prøver med en bestemt struktur som vil egne sig bedst til en af de to bulkdensitetforsøg. 

Da der ikke ses en væsentlig forskel i resultatet for bulkdensitet mellem de to forsøg, skal der andre parametre til for at afgøre om der vælges at bestemme bulkdensitetet ved archimedes eller 3D scanning. 

Struktureret lys metoden giver helt andre muligheder til fremstilling af resultater, da de bla. kan eksporteres til 3D- modeller i .pdf format, hvor prøven kan sammenlignes med et reelt billede. Dog er erfaringen fra dette projekt at 3D scanning tager lengere tid en archimedes. 

Prøvepopulationen dette forsøg er baseret på, må anses for at være lille, og der vil derfor være  behov for en analyse af en større prøvepopulation før en endelig konklusion på forskellen i bulkdensitets beregning de to forsøg i mellem. 


