\label{app:ref_obj_maalt}
\section{Terning}
Volumen er bestem ud fra gennemsnittet af bredde, lengde og højde målingerne i \vref{app:maalt_t} og bestemmes ved følgende formel:
\begin{equation}
V=l*b*h
\end{equation}
Volumen bestemmes til $124,22 \pm 0,08 cm^3$. 
Hvor standard afvigelsen findes ud fra formel:
\begin{equation}
\sigma_v=\sqrt[]{\sigma{b}^2+\sigma{l}^2+\sigma{h}^2}
\end{equation}
\begin{table}[h]
 \centering
% \caption{Terning, mål i $[mm]$, terningens dimensioner, bredden b, længden l og højden h}
  \topcaption{Mål i cm af terningens dimensioner, bredden $(b_1 -b_6)$, længden l $(h_1-h_6)$ og højden h $(h_1-h_6)$}
\label{app:maalt_t}  
\begin{tabular}{c c c c c c c c}
\toprule
b&4,99&	4,99&	4,99&	4,99&	4,98&	4,99&\\
l&	4,99&	4,99&	4,99&	5,00&	5,00&	5,00&\\
h&	5,00&	4,99&	4,99&	4,98&	4,98&	4,99&\\
\end{tabular}
\end{table}

\section{Cylinder 2}
For at bestemme volumen af cylinder 2, deles cylinderen op i fem dele, hvor der bliver tre store dele og to mindre (to hvor udfresningerne er). Volumen af hvert enkelt element bestemmes ved et gennemsnit af r1, d1 osv. 
Følgende formel anvendes:
\begin{equation}
V_i=\pi*r_i^2*h_i
\end{equation}
Det totale volumen findes ved at summere fem elementer til $132,17 \pm 0,02 cm^3$, hvor standard afvigelsen findes ved formel:
\begin{equation}
\sigma_v=\sqrt[]{\sigma{d1}^2+\sigma{r1}^2+\sigma{d2}^2+\sigma{r2}^2...+\sigma{r5}^2}	
\end{equation}

\begin{table}
 \centering
% \caption{Terning, mål i $[mm]$, terningens dimensioner, bredden b, længden l og højden h}
% \label{app:maalt_t}
 \topcaption{Mål i cm af cylinderens dimensioner, radius $(r_1,r_2,r_3,r_4,r_5)$ og højden h $(h_1,h_2,h_3,h_4,h_5)$.}
\begin{tabular}{c c c }
\multicolumn{1}{l}{} & \multicolumn{1}{c}{$[cm]$} & \multicolumn{1}{c}{$[cm]$} \\ 
\toprule
$r_1$& 2,49 & 2,49 \\
$r_2$& 1,49 & 1,49 \\ 
$r_3$& 2,50 & 2,50 \\
$r_4$& 1,50 & 1,50\\
$r_5$& 2,50 & 2,50 \\
$h_1$& 1,98 & 1,98 \\
$h_2$& 0,99 & 0,98 \\
$h_3$& 2,00 & 2,01 \\
$h_4$& 0,98 & 0,95\\
$h_5$& 2,05 & 2,05\\
\end{tabular}
\end{table}
