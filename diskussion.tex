\chapter{Diskussion af resulter og proces}
I dette kapitel diskuteres resultaterne fra de tre udførte forsøg, etterfulgt af en diskussion af fordele og ulemper ved arkimedes princip og 3D scanning til bestemmelse af bulkdensitet af permafrostkerner.

\section{Visuel- klassifikation}
Ud fra prøveemnernes klassifikation kan der ses en tendens til at de dybere prøver, i forhold til terrænet er klassificeret som Nbn, og har ingen synlig is i overfladen, derudover er de velsammenhængende. Ligeledes er der en tendens til at de prøver der er taget tættere på overfladen indeholder mere is. Ved at plotte de forskellige prøver ind i et x,y koordinatsystem ville man kunne se om der var en tendens i forhold til kompasretning og klassifikation. 

\section{3D scanning}
Resultaterne fra 3D scanning viser at der ikke er nogen betydelig forskel på de tre rotations vinkler og den afledte bulkdensitet. Forskellen på volumen bestemmelsen af referenceobjekterne ved, 3D scanning og ved manuel opmåling med skydelær, er ubetydelige taget i betragtning at volumenbestemmelsen skal anvendes til densitet bestemmelse, hvor forskellen derfor bliver minimal. 
På forhånd var det forventet at enkelte af prøveemnerne indehold nok is til at det blev en udfordring, dette gensspejles ikke i resultaterne, hvor der ikke er stor forskel at se på de prøver hvor overflade isindholdet har blevet vurderet som højt. Det kan muligvis skyldes at det ikke har været tale om klar is, og eller at der har været nok partikler imellem isinklusionerne til at scanner kunne samle ind data enkelt steder.
Ved efterbehandling af data har der været enkelte udfordringer, hvilket kan skyldes både manglende udstyr (som kraftig pc) samt undertegnedes manglende erfaring med billedanalyse. 

Tidsforbrug ved scanning varierer alt efter hvor stor rotationsvinkelen er, ved 45 graders rotationsvinkel tager det en time og fem minutter at scanne tre prøver, ved en rotationsvinkel på 72 grader tager det en time at scanne fem prøver. Det der tager længst tid ved scanning er når programmet setter de separate scans sammen, inden rotationsbordet drejer. Hvis denne funktion kan slukkes ville det være anledning til stor tidsbesparelse, derudover ville en kraftig computer kunne give anledning til tidsbesparelse ved scanning og især ved efterbehanfling af punktskyen. 

\section{Archimedes princip}
Ved densitet beregningerne i denne rapport er der anvendt en gennemsnitlig temperatur for isopar, for hvert enkelt prøveemne, som giver en fejl på 1-2\% på den endelige bulkdensitet.
Resultaterne fra densitetsbestemmelse med archimedesprincip afviger ikke væsentligt fra 3D scanning. Volumen bestemmelser fra archimedes afviger fra det volumen bestemt ved scanning, dog bliver ikke forskellen på bulkdensiteten af betydning. 

Temperatur variationen i isopar ved densitetsmålingerne er store, hvilket genspejles i den horisontale afvigelse i bilag \vref{fig:iso_dens_plot}, dette skyldes formodentligt udførelsen af forsøget. Når massen af I- isopar måles, blev den den neste prøve gjort klar, mens prøveholderen fik lov til at dryppe af over isopar beholder. Dette har medført at temperatur sensoren har vært i lufttemperatur over en lengere periode, det kan derfor se ud til at lufttemperaturen kan ha påvirket temperatur målingerne i form af at både temperatur sensor,prøven og lod har haft en anden temperatur en isoparen ved måling, det antages at prøven ikke har været nedsænket i tilstrækkelig tid for stabilisering af isopar temperaturen.     
Andre fejlkilder ved udførelse af acrkimedes princip er vægt, opstilling, aflæsning af vægt, kalibrering af temperatursensor, tab af prøvemateriale, tab af isopar som trænger ind i overfladen på prøven. 

Tidsforbruget ved archimedes er væsentlig mindre en for arkimedes princip en for 3D scanning, på 3,5 timer er der bestemt volumen tre gange på 15 prøver.
